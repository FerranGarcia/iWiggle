In general terms, the simple requirements of the project involve a hidden complexity in terms of performance and the quote 'keep it simple, no overthinking' would be a must in this case.

Specifically, the presented solution, based on Raspberry Pi, fulfils all requirements with a moderate degree of robustness. Decreasing the camera resolution in order to boost the performance as well as variable lighting conditions, make the segmentation a difficult task. It can be concluded, that using the presented approach in a more powerful board would fulfil all requirements satisfactorily.

It was experienced that working in a team composed of people with diverse set of skills and backgrounds, enriches the experience and improves the final result. Regarding to this last point, the design of a modular solution for solving the stated problem, has been successfully implemented in a way to be easily adaptable to different platforms.