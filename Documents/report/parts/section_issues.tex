\subsection{Limitations}

The hardware of the Raspberry Pi does not allow to perform any intensive
computations, so to balance between robustness and computational time is a must.

\subsection{Hardware}

\begin{itemize}
  	\item The overall performance of the Raspberry Pi showed satisfactory results for the given problem.
  	\item The reliability of the WiFi connection is low - the board is loosing the connection to WiFi dongle frequently. Loss of the connection can also be caused by an external disturbance (e.g. sharp velocity changes).
	\item RPi Camera Module is very sensitive to static electricity.
	\item Sharp increase of the torque in the motors can cause the voltage drop in
	the power supply that will lead to shut down the board.
\end{itemize}

\subsection{Software}

\begin{itemize}
  \item OpenCV takes almost 9 hours to compile.
  \item No Software is available for the GPU access.
\end{itemize}

It was necessary to synchronize the software development for the board between
the group participants. The development process was done using Microsoft Visual Studio 2013, so this led to the necessity of making the code
cross-compilable. 
Remote access to RPi is established via SSH and an xServer.
Remote Desktop Environment was provided by MobaXTerm software.