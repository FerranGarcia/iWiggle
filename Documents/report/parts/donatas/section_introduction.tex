Nowadays automation and robotics are leaving the traditional sectors as industrial manufacturing and becoming a part of daily life. It is integrating into our society, while many members still do not have a knowledge of it, since they are only familiar with the form of automation learned from massive social media.
Because of this trend of automation, it is important that educational organizations start preparing not only robotic researchers and designers, but also industrial personnel, who will be able
to take care of the robotic devices existing in our daily environment. 

A broad variety of robotic courses offered by universities or by massive open online education systems, increased during the past several years. However young students still lack opportunities to get hands-on experience on the understanding working principles, fabrication and implementation of robotic devices, which leads to the gap between theoretical and practical domain knowledge.

This project covered developing a low-cost wheeled robot, which is intended to promoting the teaching of basic computer science in schools \footnote{\url{http://www.raspberrypi.org/picademy/}}. Initial hardware design was provided by project supervisor, with limitations of only having motor servos, distance sensor, LCD screen, camera and Raspberry Pi (RPi) single-board computer.
Constraint of  Raspberry Pi computation power raised challenge to find an  optimal techniques of implementing controllers of sensors and actuators, knowing that the most of resources are consumed by image processing to implement and realise desired robot actions. Here, the optimal software design procedure is presented leading to the working example of constructed robot.
