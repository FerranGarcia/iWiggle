The present section describes the different stages of the project; design, management, software implementation and hardware improvement. It also presents the main goals of the work as well as a brief description of the main sections which compose this document.

As it has been presented in section \ref{sec:intro}, the main goal of the project is to introduce robotics in general and open-source tools such as Raspberry Pi in a dynamic and friendly manner in order to enhance the learning experience and engage children. It has been proved that the most suitable way to achieve this is to present the activity as an interactive game \cite{bransford1999people}.

Considering the importance of self-motivation in whatever successful learning process, it is important to assure an enjoyable experience for the user, conditioned by the field studied. For this reason this topic is worth it to be studied.

The main goal is to provide an independent robot platform able to interact with the user, in this case children. This idea not only forces the implementation to be as robust as possible, but also it forces to implement a strong feedback component in order to give the impression to the user what is going on in every single moment during the game development. It is so important to assure the reduction of uncertainty moments produced by an expected reaction of the system as well as a lack of information during the actions performed. 

Firstly, and according to the previous statement, a robust interaction method needs to be defined because it will provide to the user the control of the system in a reliable way. An intuitive technique to do this is to make the robot able to recognize different signs and execute the correspondent actions when one of these signs are presented. The basic signs are an arrow, a cross and a circle but it can be easily extended to more. Showing this bunch of signs to the platform, children should be able to give the appropriate instructions to move it from point A to point B.

Secondly, the concept of feedback in every single moment is essential. It will provide the current state of the process, reinforcement or additional guidance to continue the experience. In this way, two different types of feedback are defined: acoustic and visual. 

The first type of feedback is provided using a set of representative sounds indicating robot start and stop of the program (game). The second type of feedback is provided through text displayed in a LCD screen guiding the user in any situation.

The main challenge of this project is to be able to segment the sign properly under different light conditions and surfaces, but also taking into account the limited computational power of the Raspberry Pi (in this sense a comparative with another board is provided). One important improvement to accomplish successfully desired goal is the replacement of the camera installed for an IR camera. The fusion of the IR information provides a better sign recognitions under certain bad light conditions. Since the robot has to be able to move autonomously in any flat surface, it is also equipped with a proximity sensor for collision detection and two independent motors.