The list of robot's main hardware components:

\begin{itemize}
  \item Raspberry Pi Model B Board
  \item Raspberry Pi No-IR Camera
  \item 2xEMG30 12v motor with encoder and 30:1 gearbox + mounting bracket
  \item Ultrasound Proximity sensor
  \item I2C Breakout Board
\end{itemize}

Insert images of the components (?).

\subsection*{Mechanical Part}

The robot is built on a metal frame with two driving wheels mounted on a EMG30
gearbox connected to the frame using the gearbox bracket. A metal ball wheel
located in front of the robot provides additional support. The frame consists of
metal profiles with a perforated plate that serves as a base for mounting
electronic components.

\subsection*{Electronics}

The main electronic component of the robot is the Raspberry Pi single-board
computer. Its technical specifications are listed in the Table
\ref{tab:rpi_specs}.

\begin{table}[h!]
	\setlength\extrarowheight{2pt}
    \begin{tabularx}{\textwidth}{XX}
    
    ~                 & ~                                                           \\
    System on a chip  & Broadcom BCM2835                                            \\
    CPU               & 700 MHz ARM11                                               \\
    GPU               & Broadcom VideoCore IV @ 250 MHz                             \\
    RAM               & 512 Mb shared with GPU                                      \\
    Memory            & SD Card Slot                                                \\
    Ports             & 2x USB, LAN, 3.5mm phone jack, HDMI,  GPIO (UART, I2C, SPI) \\
    Power Consumption & 700 mA (3.5 W)                                              \\
    
    \end{tabularx}
    \caption{Raspberry Pi Model B Specifications}
    \label{tab:rpi_specs}
\end{table}


The board is powered using 11.1V Turnigy LiPo Battery via power adapter that
splits the voltage into 12V (motors) and 5V lines. According to the external
power supply measurements, the average total consumption during movement is
(insert here).

The Raspberry Pi peripherals include a WiFi adapter, GPIO Breakout board for
simplifying connection, and a Raspberry Pi camera. RPi camera is a
board-specific 5Mp camera module connected via 15-pin MIPI (Mobile Industry
Processor Interface) connector. It is capable of capturing 1080p video and
perform basic hardware video processing. Two different types of cameras were
tested - standard camera and near-IR ``NoIR'' camera with the absence of
infrared filter. The performance of these cameras compared to standard USB
Webcam is shown in the Table \ref{tab:cam_perf}. In order to
increase the reliability and robustness of NoIR camera, a couple of IR LEDs were
installed to provide additional illumination.

Talk about sign design for the near-IR camera here.

% Fix table here
\begin{table}[h!]
	\setlength\extrarowheight{2pt}
	\setlength\arraycolsep{5pt}
    \begin{tabularx}{\textwidth}{XXXX}
    Resolution / Camera Setup & Raspberry Pi  (CPU) + Raspberry Pi Camera &
    Raspberry Pi (CPU) + Logitech C270 & Odroid-U3 (CPU) + Logitech C270 \\
    160x120            & 18           & 9          & 30         \\
    320x240            & 6            & 4.8        & 20         \\
    640x480            & 2.5          & 2.3        & 11         \\
    1280x720           & 0.6          & 0.8        & 4          \\
    \end{tabularx}
    \caption{Camera Performance comparison}
    \label{tab:cam_perf}
\end{table}

As can be seen from the Table \ref{tab:cam_perf}, the main advantage of using
RPi camera is its performance - it outperforms generic Webcam grace to hardware
optimization and, in particular, absence of resizing routine in the code.
However, the disadvantage lies in the specificity of the camera, that can only
be used on the RPi board, while the USB Webcam can be used everywhere, so in
order to port the code to another platform, the camera change as well as some
code modification are required.

The robot is also equipped with an ultrasound proximity sensor used for obstacle
detection.

An I2C LCD Screen module provides an interface for visual feedback.